\documentclass[a4paper,11pt]{article}
\usepackage[utf8]{inputenc}
\usepackage[italian]{babel}
\usepackage{geometry}
\usepackage{array}
\usepackage{enumitem}
\usepackage{booktabs}
\usepackage{xcolor}
\usepackage{listings}
\usepackage{hyperref}
\usepackage[T1]{fontenc}
\usepackage{tgbonum}
\usepackage{graphicx}
\graphicspath{ {./img/} }
\author{Spazzali Filippo IN0501237}

\geometry{margin=2.5cm}

\definecolor{lightgray}{rgb}{0.9,0.9,0.9}
\definecolor{darkblue}{rgb}{0.0,0.0,0.6}

\lstdefinestyle{SQL}{
  language=SQL,
  basicstyle=\ttfamily\small,
  keywordstyle=\color{blue}\bfseries,
  stringstyle=\color{red},
  commentstyle=\color{gray},
  identifierstyle=\color{teal},
  columns=flexible,
  tabsize=2,
  breaklines=true,
  showstringspaces=false,
  morecomment=[l]{--},
  morecomment=[s]{/*}{*/},
  morekeywords={CREATE, DATABASE, TABLE, DROP, IF, EXISTS, USE, PRIMARY, KEY, FOREIGN, REFERENCES, CONSTRAINT, DELIMITER, SIGNAL, SELECT, INTO, DECLARE, BEGIN, END, PROCEDURE, CALL, JOIN, INNER, LEFT, ON, UPDATE, CASCADE, COUNT, DISTINCT, GROUP_CONCAT, ORDER, BY, WHERE, BETWEEN, DATE_SUB, CURDATE, DOUBLE, BEFORE, FOR, EACH, ROW, CONTINUE, HANDLER, CURSOR, WHILE, DO, FETCH, OPEN, INOUT, OUT},
  sensitive=false
}

\title{\textbf{Progetto di Basi di Dati\\``Studio di Modellazione 3D''}}
\date{}

\begin{document}
{\fontfamily{phv}\selectfont
\maketitle

\tableofcontents
\newpage

\section{Presentazione Progetto}
Uno studio professionale di modellazione 3D gestisce vari progetti per diversi clienti, tenendo traccia di tutti i lavori svolti, i modelli creati e le risorse utilizzate. Lo studio impiega diversi artisti specializzati in varie aree della modellazione 3D.

Lo studio desidera mantenere informazioni dettagliate su ogni cliente, che includono ragione sociale, partita IVA, indirizzo (città, via, civico), email, telefono e, facoltativamente, sito web. I clienti possono essere sia aziende che privati.

Per quanto riguarda i dipendenti dello studio, si tiene traccia del nome, cognome, codice fiscale, data e luogo di nascita, email, telefono e specializzazione (modellazione, texturing, rigging, animazione, illuminazione, rendering).

Ogni progetto commissionato allo studio ha un codice identificativo, un titolo, una data di inizio, una data di consegna prevista, un budget concordato, e appartiene a una categoria (architettura, videogiochi, cinema, pubblicità, ingegneria, formazione). Ogni progetto è assegnato a un cliente e può coinvolgere più dipendenti con ruoli diversi.

I progetti si compongono di modelli 3D, ciascuno con un proprio codice identificativo, nome, data di creazione, tipo (organico, inorganico, ambiente, personaggio), poligoni totali e dimensione in megabyte. Ogni modello può utilizzare diverse risorse esterne come texture, che hanno un proprio identificativo, nome, risoluzione, dimensione in megabyte e formato (JPG, PNG, TIFF, EXR).

Lo studio utilizza vari software di modellazione 3D, di cui si tiene traccia del nome, versione, sviluppatore, costo della licenza e data di scadenza della licenza. Per ogni progetto e modello, si registra quali software sono stati utilizzati.

Inoltre, lo studio offre servizi di formazione attraverso corsi di modellazione 3D solo ed esclusivamente ai propri clienti (ovvero persone che hanno già richiesto un progetto). Ogni corso ha un codice, titolo, descrizione, durata in ore, livello (base, intermedio, avanzato) e numero massimo di partecipanti. I corsi sono tenuti da dipendenti dello studio e possono essere frequentati solo da clienti già esistenti, di cui si mantengono nome, cognome, email, telefono e data di iscrizione. SI specifica che i corsi sono gratuiti.

\subsection*{Azioni}
\begin{itemize}
  \item L'amministrazione vuole ottenere la lista dei progetti completati nell'ultimo mese per la 
  
  fatturazione 
  \begin{flushright}
(2 Volte al Mese)
  \end{flushright}
  \item Il responsabile tecnico vuole ottenere le statistiche sull'utilizzo dei software per pianificare il rinnovo delle licenze 
  \begin{flushright}
  (1 Volta ogni 3 Mesi)
  \end{flushright}
  \item Il responsabile marketing vuole ottenere i dati dei clienti per inviare promozioni sui corsi di formazione 
  \begin{flushright}
  (1 Volta al Mese)
  \end{flushright}
  \item Il direttore artistico vuole ottenere le informazioni sui modelli più complessi (con più poligoni) per ottimizzare le risorse hardware 
  \begin{flushright}
  (1 Volta al Mese)
  \end{flushright}
\end{itemize}
\newpage


\section{Schema Entity-Relationship}
\includegraphics[scale=0.35]{ER1.png}
\newpage


\section{Dizionario dei Dati}

\subsection{Dizionario dei Dati - Entità}

\begin{table}[h]
\centering
\begin{tabular}{|p{2cm}|p{4cm}|p{6cm}|p{4cm}|}
\hline
\textbf{Entità} & \textbf{Descrizione} & \textbf{Attributi} & \textbf{Identificatore} \\
\hline
Cliente & cliente dello studio & partita IVA, ragione sociale, email, telefono, indirizzo, sito web & partita IVA \\
\hline
Dipendente & dipendente dello studio & codice fiscale, nome, cognome, data di nascita, telefono, email, specializzazione & codice fiscale \\
\hline
Progetto & progetto commissionato allo studio & codice progetto, titolo, data inizio, data consegna, budget, categoria & codice progetto \\
\hline
Modello & modello 3D creato per un progetto & codice modello, nome, data creazione, tipo, poligoni, dimensione & codice modello \\
\hline
Risorsa & risorsa utilizzata nei modelli & id risorsa, nome, dimensione, formato & id risorsa \\
\hline
Software & software utilizzato per modellazione & nome, versione, sviluppatore, costo licenza, scadenza licenza & nome, versione \\
\hline
Corso & corso di formazione offerto & codice corso, titolo, descrizione, durata, prezzo, livello, max partecipanti & codice corso \\
\hline
\end{tabular}
\caption{Dizionario dei Dati - Entità}
\end{table}
\newpage


\subsection{Dizionario dei Dati - Relazioni}

\begin{table}[h]
\centering
\begin{tabular}{|p{3.5cm}|p{4cm}|p{5cm}|p{3cm}|}
\hline
\textbf{Relazione} & \textbf{Descrizione} & \textbf{Composizione} & \textbf{Attributi} \\
\hline
Commissione & cliente che commissiona un progetto & partita IVA cliente, codice progetto & - \\
\hline
Assegnazione & dipendente assegnato a un progetto & codice fiscale dipendente, codice progetto & ore lavorate, ruolo \\
\hline
Composizione & modello appartenente a un progetto & codice modello, codice progetto & - \\
\hline
Utilizzo Risorsa & risorsa utilizzata in un modello & id risorsa, codice modello & - \\
\hline
Utilizzo Software & software utilizzato per un modello & nome software, versione software, codice modello & - \\
\hline
Insegnamento & dipendente che insegna un corso & codice fiscale dipendente, codice corso & - \\
\hline
Partecipazione & cliente che partecipa a un corso & PartitaIVA, codice corso & data iscrizione \\
\hline
\end{tabular}
\caption{Dizionario dei Dati - Relazioni}
\end{table}

\section{Vincoli Non Esprimibili}
Dall'analisi del testo del problema emergono i seguenti vincoli non esprimibili:
\begin{itemize}
  \item I corsi non possono avere più iscritti del numero massimo di partecipanti specificato.
  \item I progetti devono avere almeno un dipendente assegnato con ruolo di project manager.
  \item Le texture utilizzate nei modelli devono avere una risoluzione adeguata rispetto alla complessità del modello.
\end{itemize}
\newpage


\section{Tabella dei Volumi}

\begin{table}[h]
\centering
\begin{tabular}{|p{3cm}|p{1.5cm}|p{2cm}|}
\hline
\textbf{CONCETTO} & \textbf{TIPO} & \textbf{VOLUME} \\
\hline
cliente & E & 100 \\
\hline
dipendente & E & 30 \\
\hline
progetto & E & 200 \\
\hline
modello & E & 1000 \\
\hline
risorsa & E & 5000 \\
\hline
software & E & 15 \\
\hline
corso & E & 20 \\
\hline
commissione & R & 200 \\
\hline
assegnazione & R & 600 \\
\hline
composizione & R & 1000 \\
\hline
utilizzo\_risorsa & R & 10000 \\
\hline
utilizzo\_software & R & 2000 \\
\hline
insegnamento & R & 20 \\
\hline
partecipazione & R & 1000 \\
\hline
\end{tabular}
\caption{Tabella dei Volumi}
\end{table}

\section{Schema Entity-Relationship Ristrutturato}
\subsection{Eliminazione generalizzazioni}
Per quanto riguarda l'entità cliente, si è scelto di accorpare la generalizzazione azienda/privato nell'entità genitore, aggiungendo l'attributo "tipo" che può assumere i valori "azienda" o "privato". Questo approccio è stato scelto perché da un'analisi delle azioni richieste sulla base di dati risulta che gli accessi a tale entità sono contestuali e non si richiedono frequentemente i record di una sola specializzazione piuttosto che l'altra.

\subsection{Analisi ridondanze}
La relazione "assegnazione" presenta l'attributo "ore lavorate". Questo attributo, pur essendo molto importante per le statistiche richieste dal direttore dello studio, viene eliminato in quanto campo calcolabile.

\subsection{Eliminazione attributi composti}
Vengono inoltre scomposti gli attributi composti in attributi semplici, come "indirizzo" in città, via, civico e CAP, e "dimensione" nella tabella modello in larghezza, altezza e profondità.
\newpage


\subsection{Scelta Identificatori Primari}
Per ogni entità è stato evidenziato un identificatore primario. Rispettivamente, sono stati scelti:

\begin{itemize}
  \item Cliente $\longrightarrow$ la partita IVA;
  \item Dipendente $\longrightarrow$ il codice fiscale;
  \item Progetto $\longrightarrow$ il codice progetto;
  \item Modello $\longrightarrow$ il codice modello;
  \item Risorsa $\longrightarrow$ l'id risorsa;
  \item Software $\longrightarrow$ la combinazione di nome e versione;
  \item Corso $\longrightarrow$ il codice corso;
\end{itemize}


\includegraphics[scale=0.35]{ER2.png}
\newpage


\section{Schema Logico}

\begin{itemize}
  \item Cliente (PartitaIVA, RagioneSociale, Email, Telefono, SitoWeb, Tipo, Città, Via, Civico, CAP)
  \item Dipendente (CodiceFiscale, Nome, Cognome, DataNascita, Telefono, Email, Specializzazione)
  \item Progetto (CodiceProgetto, Titolo, DataInizio, DataConsegna, Budget, Categoria, IdCliente)
  \item Modello (CodiceModello, Nome, DataCreazione, Tipo, Poligoni, Larghezza, Altezza, Profondità, IdProgetto, nomeSoftwsre, versioneSoftware, IdRisorsa)
  \item Risorsa (IdRisorsa, Nome, Dimensione, Formato)
  \item Software (Nome, Versione, Sviluppatore, CostoLicenza, ScadenzaLicenza)
  \item Corso (CodiceCorso, Titolo, Descrizione, Durata, Prezzo, Livello, MaxPartecipanti, IdInsegnante)
  \item Assegnazione (IdAssegnazione, IdDipendente, IdProgetto, Ruolo)
   \item Partecipazione (IdPartecipazione, IdIscritto, IdCorso, DataIscrizione)

\end{itemize}

\includegraphics[scale=0.4]{tabelle.png}

\newpage

\subsection{Normalizzazione}
\begin{itemize}
  \item La base di dati è già in Prima Forma Normale, tutte le colonne sono atomiche.
  \item La base di dati è già in Seconda Forma Normale, ciascuna colonna dipende dalla primary key.
  \item La base di dati è già in Terza Forma Normale, ogni attributo dipende solo dalla primary key.
\end{itemize}

\newpage

\section{Query SQL per la Realizzazione della Base di Dati}

\begin{lstlisting}[style=SQL]
CREATE DATABASE IF NOT EXISTS `studiomodellazione3d`;
USE `studiomodellazione3d`;

DROP TABLE IF EXISTS `cliente`;
CREATE TABLE `cliente` (
  `partitaIVA` VARCHAR(15) NOT NULL,
  `ragioneSociale` VARCHAR(100) NOT NULL,
  `email` VARCHAR(100) NOT NULL,
  `telefono` INT(20) NOT NULL,
  `sitoWeb` VARCHAR(100) DEFAULT NULL,
  `tipo` ENUM('azienda','privato') NOT NULL,
  `citta` VARCHAR(50) NOT NULL,
  `via` VARCHAR(100) NOT NULL,
  `civico` INT(10) NOT NULL,
  `cap` INT(5) NOT NULL,
  
  PRIMARY KEY (`partitaIVA`)
);

DROP TABLE IF EXISTS `dipendente`;
CREATE TABLE `dipendente` (
  `codiceFiscale` VARCHAR(16) NOT NULL,
  `nome` VARCHAR(50) NOT NULL,
  `cognome` VARCHAR(50) NOT NULL,
  `dataNascita` DATE NOT NULL,
  `telefono` INT(20) NOT NULL,
  `email` VARCHAR(100) NOT NULL,
  `specializzazione` ENUM('modellazione','texturing','rigging','animazione','illuminazione','rendering') NOT NULL,
  
  PRIMARY KEY (`codiceFiscale`)
);

DROP TABLE IF EXISTS `progetto`;
CREATE TABLE `progetto` (
  `codiceProgetto` INT(10) NOT NULL,
  `titolo` VARCHAR(100) NOT NULL,
  `dataInizio` DATE NOT NULL,
  `dataConsegna` DATE NOT NULL,
  `budget` INT(10) NOT NULL,
  `idCliente` INT(15) NOT NULL,
  
  PRIMARY KEY (`codiceProgetto`),
  FOREIGN KEY (`idCliente`)
  REFERENCES `cliente` (`partitaIVA`)
);
\end{lstlisting}
\newpage
\begin{lstlisting}[style=SQL]
DROP TABLE IF EXISTS `modello`;
CREATE TABLE `modello` (
  `codiceModello` INT(10) NOT NULL,
  `nome` VARCHAR(100) NOT NULL,
  `dataCreazione` DATE NOT NULL,
  `tipo` ENUM('organico','inorganico','ambiente','personaggio') NOT NULL,
  `poligoni` INT NOT NULL,
  `larghezza` DOUBLE NOT NULL,
  `altezza` DOUBLE NOT NULL,
  `profondita` DOUBLE NOT NULL,
  `idProgetto` INT(10) NOT NULL,
  `nomeSoftware` VARCHAR(100) NOT NULL,
  `versioneSoftware` VARCHAR(100) NOT NULL,
  `idRisorsa` INT(10) NOT NULL,
  
  PRIMARY KEY (`codiceModello`),
  FOREIGN KEY (`idProgetto`)
  REFERENCES `progetto` (`codiceProgetto`),
  FOREIGN KEY (`nomeSoftware`, `versioneSoftware`)
  REFERENCES `software` (`nome`, `versione`),
  FOREIGN KEY (`idRisorsa`)
  REFERENCES `risorsa` (`idRisorsa`)
);

DROP TABLE IF EXISTS `risorsa`;
CREATE TABLE `risorsa` (
  `idRisorsa` INT(10) NOT NULL,
  `nome` VARCHAR(100) NOT NULL,
  `dimensione` INT NOT NULL,
  `formato` ENUM('JPG','PNG','TIFF','EXR') NOT NULL,

  PRIMARY KEY (`idRisorsa`)
);

DROP TABLE IF EXISTS `software`;
CREATE TABLE `software` (
  `nome` VARCHAR(100) NOT NULL,
  `versione` VARCHAR(100) NOT NULL,
  `sviluppatore` VARCHAR(100) NOT NULL,
  `costoLicenza` INT(10) NOT NULL,
  `scadenzaLicenza` DATE NOT NULL,
  
  PRIMARY KEY (`nome`,`versione`)
);
\end{lstlisting}
\newpage
\begin{lstlisting}[style=SQL]
DROP TABLE IF EXISTS `corso`;
CREATE TABLE `corso` (
  `codiceCorso` INT(10) NOT NULL,
  `titolo` VARCHAR(100) NOT NULL,
  `descrizione` VARCHAR(100) NOT NULL,
  `durata` INT(10) NOT NULL,
  `prezzo` DOUBLE NOT NULL,
  `livello` ENUM('base','intermedio','avanzato') NOT NULL,
  `maxPartecipanti` INT(10) NOT NULL,
  `idInsegnate` VARCHAR(16) NOT NULL,
  
  PRIMARY KEY (`codiceCorso`),
  FOREIGN KEY (`idInsegnate`)
  REFERENCES `dipendente` (`codiceFiscale`)
);

DROP TABLE IF EXISTS `assegnazione`;
CREATE TABLE `assegnazione` (
  `idAssegnazione` INT(10) NOT NULL,
  `idDipendente` INT(10) NOT NULL,
  `idProgetto` INT(10) NOT NULL,
  `ruolo` VARCHAR(50) NOT NULL,
  
  PRIMARY KEY (`idAssegnazione`),
  FOREIGN KEY (`idDipendente`)
  REFERENCES `dipendente` (`codiceFiscale`),
  FOREIGN KEY (`idProgetto`)
  REFERENCES `progetto` (`codiceProgetto`)
);

DROP TABLE IF EXISTS `partecipazione`;
CREATE TABLE `partecipazione` (
  `idPartecipazione` INT(10) NOT NULL,
  `idCliente` INT(10) NOT NULL,
  `idCorso` INT(10) NOT NULL,
  `dataIscrizione` DATE NOT NULL,
  
  PRIMARY KEY (`idPartecipazione`),
  FOREIGN KEY (`idCliente`)
  REFERENCES `cliente` (`partitaIVA`)
  FOREIGN KEY (`idCorso`)
  REFERENCES `corso` (`codiceCorso`)
);
\end{lstlisting}
\newpage

\section{Query SQL Aggiuntive}

\subsection{Trigger per controllare che non vengano iscritti più partecipanti del numero massimo per un corso}

\begin{lstlisting}[style=SQL][caption=Trigger per il controllo del numero massimo di partecipanti]
DELIMITER $$
CREATE TRIGGER trg_beforeInsertPartecipazione BEFORE INSERT ON partecipazione
FOR EACH ROW
BEGIN
  DECLARE numIscritti INT;
  DECLARE maxIscritti INT;
  
  SELECT COUNT(*) INTO numIscritti FROM partecipazione WHERE idCorso = NEW.idCorso;
  SELECT maxPartecipanti INTO maxIscritti FROM corso WHERE codiceCorso = NEW.idCorso;
  
  IF (numIscritti >= maxIscritti) THEN
    SIGNAL SQLSTATE '45001' SET MESSAGE_TEXT = 'Numero massimo di partecipanti raggiunto per questo corso';
  END IF;
END $$
DELIMITER ;
\end{lstlisting}

\subsection{Trigger per controllare che un project manager sia sempre assegnato ad ogni progetto}

\begin{lstlisting}[style=SQL][caption=Trigger per il controllo della presenza di un project manager]
DELIMITER $$
CREATE TRIGGER trg_beforeUpdateAssegnazione BEFORE UPDATE ON assegnazione
FOR EACH ROW
BEGIN
  DECLARE pmCount INT;
  
  IF (OLD.ruolo = 'project manager' AND NEW.ruolo != 'project manager') THEN
    SELECT COUNT(*) INTO pmCount FROM assegnazione 
    WHERE idProgetto = NEW.idProgetto AND ruolo = 'project manager' AND idDipendente != NEW.idDipendente;
    
    IF (pmCount = 0) THEN
      SIGNAL SQLSTATE '45002' SET MESSAGE_TEXT = 'Ogni progetto deve avere almeno un project manager';
    END IF;
  END IF;
END $$
DELIMITER ;
\end{lstlisting}
\newpage

\subsection{Vista per ottenere il carico di lavoro dei dipendenti nei progetti attivi}

\begin{lstlisting}[style=SQL][caption=Stored Procedure per il carico di lavoro dei dipendenti]
CREATE VIEW v_caricoLavoroDipendenti AS
SELECT 
  d.nome, 
  d.cognome, 
  d.specializzazione, 
  p.titolo AS progetto, 
  a.ruolo
FROM dipendente d
INNER JOIN assegnazione a ON d.codiceFiscale = a.idDipendente
INNER JOIN progetto p ON a.idProgetto = p.codiceProgetto
WHERE p.dataConsegna >= CURDATE()
ORDER BY d.cognome, d.nome, p.titolo;

SELECT * FROM v_caricoLavoroDipendenti;
\end{lstlisting}

\subsection{Stored Procedure per ottenere i progetti completati nell'ultimo mese}

\begin{lstlisting}[style=SQL][caption=Stored Procedure per i progetti completati nell'ultimo mese]
DELIMITER $$
CREATE PROCEDURE sp_getProgettiCompletatiUltimoMese()
BEGIN
  SELECT 
    p.codiceProgetto,
    p.titolo,
    p.dataInizio,
    p.dataConsegna,
    p.budget,
    c.ragioneSociale AS cliente,
    c.partitaIVA
  FROM progetto p
  INNER JOIN cliente c ON p.clienteID = c.partitaIVA
  WHERE p.dataConsegna BETWEEN DATE_SUB(CURDATE(), INTERVAL 1 MONTH) AND CURDATE()
  ORDER BY p.dataConsegna DESC;
END $$
DELIMITER ;

CALL sp_getProgettiCompletatiUltimoMese();
\end{lstlisting}

\newpage
\subsection{Vista per ottenere statistiche sull'utilizzo dei software}

\begin{lstlisting}[style=SQL][caption=Stored Procedure per le statistiche sull'utilizzo dei software]
CREATE VIEW v_statisticheUtilizzoSoftware AS
SELECT 
  s.nome,
  s.versione,
  s.sviluppatore,
  s.scadenzaLicenza,
  COUNT(DISTINCT us.idModello) AS numeroModelli,
  COUNT(DISTINCT m.idProgetto) AS numeroProgetti
FROM software s
LEFT JOIN utilizzoSoftware us ON s.nome = us.softwareNome AND s.versione = us.softwareVersione
LEFT JOIN modello m ON us.idModello = m.codiceModello
GROUP BY s.nome, s.versione
ORDER BY numeroModelli DESC;

SELECT * FROM v_statisticheUtilizzoSoftware;
\end{lstlisting}

\subsection{Vista per ottenere i dati dei clienti per promozioni}

\begin{lstlisting}[style=SQL][caption=Stored Procedure per i dati dei clienti per promozioni]
CREATE VIEW v_datiClientiPerPromozioni AS
SELECT 
  c.partitaIVA,
  c.ragioneSociale,
  c.email,
  c.telefono,
  COUNT(DISTINCT p.codiceProgetto) AS numeroProgetti,
  MAX(p.dataConsegna) AS ultimoProgetto
FROM cliente c
LEFT JOIN progetto p ON c.partitaIVA = p.clienteID
GROUP BY c.partitaIVA
ORDER BY numeroProgetti DESC;

SELECT * FROM v_datiClientiPerPromozioni;
\end{lstlisting}

\newpage
\subsection{Vista per ottenere i modelli più complessi}

\begin{lstlisting}[style=SQL][caption=Stored Procedure per i modelli più complessi]
CREATE VIEW v_modelliPiuComplessi AS
SELECT 
  m.codiceModello,
  m.nome,
  m.tipo,
  m.poligoni,
  m.dimensione,
  p.titolo AS progetto,
  GROUP_CONCAT(DISTINCT CONCAT(s.nome, ' ', s.versione)) AS software
FROM modello m
INNER JOIN progetto p ON m.idProgetto = p.codiceProgetto
LEFT JOIN utilizzoSoftware us ON m.codiceModello = us.idModello
LEFT JOIN software s ON us.softwareNome = s.nome AND us.softwareVersione = s.versione
GROUP BY m.codiceModello
ORDER BY m.poligoni DESC
LIMIT 20;

SELECT * FROM v_modelliPiuComplessi;

CALL sp_getModelliPiuComplessi();
\end{lstlisting}

\subsection{Stored Procedure per ottenere una lista di email di tutti i clienti, per scopi promozionali}

\begin{lstlisting}[style=SQL][caption=Stored Procedure per la lista di email dei clienti]
DELIMITER $$
CREATE PROCEDURE sp_listaEmail(OUT listaemail VARCHAR(10000))
BEGIN
  DECLARE finito INTEGER DEFAULT 0;
  DECLARE var_email VARCHAR(100) DEFAULT "";
  DECLARE email_cursor CURSOR FOR SELECT email FROM cliente;
  DECLARE CONTINUE HANDLER FOR NOT FOUND SET finito = 1;
  
  OPEN email_cursor;
  
  WHILE (finito = 0) DO
    FETCH email_cursor INTO var_email;
    IF finito = 0 THEN
      SET listaemail = CONCAT(var_email, "; ", listaemail);
    END IF;
  END WHILE;
  
  CLOSE email_cursor;
END $$
DELIMITER ;

SET @lista_emails = '';
CALL sp_listaEmail(@lista_emails);

\end{lstlisting}
}
\end{document}
